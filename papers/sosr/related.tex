\section{Related Work}

Click \cite{click} is an architecture for a reconfigurable dataplane pipeline.

NetKat\cite{netkat}

Concurrent NetCore\cite{cnetcore}

P4\cite{p4} offers a high level

Protocol Oblivious Forwarding\cite{pof}

Wripe \cite{wripe}

TinyNBI \cite{tinynbi}

Mininet\cite{mininet}

Visual Network Description \cite{vnd}

NCTUns \cite{nctuns}

Viz
------
There has been work [x,y,z] regarding network simulators, but this prior work focuses on observing packets at the topology level. Network simulators x, y, z visualize the forwarding path of a packet as it traverses a network topology. Visualzing a packet that is forwarded out the wrong port of a router, an observer could infer an incorrect routing table. As traditional network devices are replaced by programmable dataplanes, packet path visualizations at the topology level become less helpful.

Level of observation
---------------------
The ability to observe the functionality of a programmable dataplane is 
critical to ensuring the correctness of a sdn application. 
Recent tools [x, y, z] give users the ability to observe software defined
networks, but only at the topology level....  

NCTUns \cite{nctuns} is network simulator is capable, through the use of linux
containers, of simulating multiple nodes on a single host. Because each 
node is just a virtualized OS users are able to experiment with real 
implementations of network stacks, dataplanes, and controllers. 

Mininet \cite{mininet} eliminates the burden of dealing with network namespaces 
by providing an API for orchestrating the configuration of a topology, and 
modeling their namespace configuration in terms of network elements (hosts, 
switches, links, controllers). Visual Network Description \cite{vnd} is a 
GUI for configuring the hosts, controllers, and switches in a mininet
topology.

While the ability to bring any software implementation is great... not all
programmable dataplanes are implemented in software... Most software
implementations of dataplanes do not provide a method to simulate
the configuration of its capabilities.  
