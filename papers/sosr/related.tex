\section{Related Work}

% Can we start categorizing the related work into:
% - parametric pipelines
%   - click
%   - pof?
% - types and constraints over pipeline abstractions
%   - p4
%   - concurrent netcore
%   - netkat?
% - pipeline emulators/simulators
%   - dp verification 
% - pipeline visualization

\textbf{Parametric Pielines} - Click \cite{click} is language and architecture 
for building a parametric packet processing pipeline. A Click pipeline is 
composed of many individual packet processing stages. Click uses a single 
abstraction, an 'element', to define the behavior of each stage in the pipeline.
Click provides a declarative configuration language that is used to define
how pipeline elements are connected together. The configuration of a click
pipline as a whole can be analyzed to ensure systemwide properties are
satisfied. Individual elements must be analyzed manually as their semantics
are not clear. 

\textbf{Types and Constraints}
Generally all packet processors must be able to do the 
following: extract headers from packet data, select a table based on extracted 
headers, and perform actions associated with a table entry. P4\cite{p4} is a 
high level dataplane configuration language which aims to
achieve protocol and platform independence by generalizing a dataplane 
forwarding model and defining a set of types common to packet processors that 
each stage in a pipeline can operate on: headers, parsers, tables, actions, and
control programs. The header type specifys the structural and semantic 
constraints of a valid protocol header message format. Parsers define a valid 
sequence of header types. The tables type specifys what header types to match 
on and what actions to perform on the packet associated with the matched header.

NetKat\cite{netkat}
Concurrent NetCore\cite{cnetcore}
Protocol Oblivious Forwarding\cite{pof}
Wripe \cite{wripe}
TinyNBI \cite{tinynbi}

\textbf{Pipeline Simulators \& Emulators} -
Dataplane verification


Mininet\cite{mininet}
Visual Network Description \cite{vnd}
NCTUns \cite{nctuns}

Viz
------
There has been work [x,y,z] regarding network simulators, but this prior work focuses on observing packets at the topology level. Network simulators x, y, z visualize the forwarding path of a packet as it traverses a network topology. Visualzing a packet that is forwarded out the wrong port of a router, an observer could infer an incorrect routing table. As traditional network devices are replaced by programmable dataplanes, packet path visualizations at the topology level become less helpful.

Level of observation
---------------------
The ability to observe the functionality of a programmable dataplane is 
critical to ensuring the correctness of a sdn application. 
Recent tools [x, y, z] give users the ability to observe software defined
networks, but only at the topology level....  

NCTUns \cite{nctuns} is network simulator is capable, through the use of linux
containers, of simulating multiple nodes on a single host. Because each 
node is just a virtualized OS users are able to experiment with real 
implementations of network stacks, dataplanes, and controllers. 

Mininet \cite{mininet} eliminates the burden of dealing with network namespaces 
by providing an API for orchestrating the configuration of a topology, and 
modeling their namespace configuration in terms of network elements (hosts, 
switches, links, controllers). Visual Network Description \cite{vnd} is a 
GUI for configuring the hosts, controllers, and switches in a mininet
topology.

While the ability to bring any software implementation is great... not all
programmable dataplanes are implemented in software... Most software
implementations of dataplanes do not provide a method to simulate
the configuration of its capabilities.  
