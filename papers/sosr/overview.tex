\section{Overview}



% OpenFlow Experimentation & Debugging
% How to make it easy?
%   What makes it hard?
% How to make it useful?
%   Why is it not useful in current state?

\textbf{Contributions.} This paper makes a series of contributions that help
with: learning, extending, and debugging the OpenFlow data plane configurations.
\begin{itemize}
  \item \textbf{Parametric pipeline} - The overall pipeline abstraction is
        highly configurable allowing configurations that model OpenFlow 
        versions 1.0, 1.1, 1.2, 1.3, 1.4, and beyond.
  \item \textbf{Definable protocols} - Protocol abstractions are defined 
        independent of the pipeline. Table match and protocol modification are
        built-in operations and synthesized for any well defined protocol.
  \item \textbf{Table types and constraints} - Flow tables carry types and
        constraints on flow entries that prevent users from creating certain
        unsafe behaviors. The types and constraints can also be used to model
        specific vendor implementations of the OpenFlow pipeline.
  \item \textbf{Single step simulation} - A pipeline execution can be simulated
        against a trace of input packets A single step correlates to OpenFlow
        actions.
  \item \textbf{Abstraction visualization} - There are graphical representations
        of most of the abstractions of the system. This provides a more intuitive
        view of the system than blah.
\end{itemize}
